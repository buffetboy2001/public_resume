% Options for packages loaded elsewhere
\PassOptionsToPackage{unicode}{hyperref}
\PassOptionsToPackage{hyphens}{url}
\PassOptionsToPackage{dvipsnames,svgnames,x11names}{xcolor}
%
\documentclass[
  letterpaper,
  DIV=11,
  numbers=noendperiod]{scrartcl}

\usepackage{amsmath,amssymb}
\usepackage{iftex}
\ifPDFTeX
  \usepackage[T1]{fontenc}
  \usepackage[utf8]{inputenc}
  \usepackage{textcomp} % provide euro and other symbols
\else % if luatex or xetex
  \usepackage{unicode-math}
  \defaultfontfeatures{Scale=MatchLowercase}
  \defaultfontfeatures[\rmfamily]{Ligatures=TeX,Scale=1}
\fi
\usepackage{lmodern}
\ifPDFTeX\else  
    % xetex/luatex font selection
\fi
% Use upquote if available, for straight quotes in verbatim environments
\IfFileExists{upquote.sty}{\usepackage{upquote}}{}
\IfFileExists{microtype.sty}{% use microtype if available
  \usepackage[]{microtype}
  \UseMicrotypeSet[protrusion]{basicmath} % disable protrusion for tt fonts
}{}
\makeatletter
\@ifundefined{KOMAClassName}{% if non-KOMA class
  \IfFileExists{parskip.sty}{%
    \usepackage{parskip}
  }{% else
    \setlength{\parindent}{0pt}
    \setlength{\parskip}{6pt plus 2pt minus 1pt}}
}{% if KOMA class
  \KOMAoptions{parskip=half}}
\makeatother
\usepackage{xcolor}
\setlength{\emergencystretch}{3em} % prevent overfull lines
\setcounter{secnumdepth}{-\maxdimen} % remove section numbering
% Make \paragraph and \subparagraph free-standing
\ifx\paragraph\undefined\else
  \let\oldparagraph\paragraph
  \renewcommand{\paragraph}[1]{\oldparagraph{#1}\mbox{}}
\fi
\ifx\subparagraph\undefined\else
  \let\oldsubparagraph\subparagraph
  \renewcommand{\subparagraph}[1]{\oldsubparagraph{#1}\mbox{}}
\fi


\providecommand{\tightlist}{%
  \setlength{\itemsep}{0pt}\setlength{\parskip}{0pt}}\usepackage{longtable,booktabs,array}
\usepackage{calc} % for calculating minipage widths
% Correct order of tables after \paragraph or \subparagraph
\usepackage{etoolbox}
\makeatletter
\patchcmd\longtable{\par}{\if@noskipsec\mbox{}\fi\par}{}{}
\makeatother
% Allow footnotes in longtable head/foot
\IfFileExists{footnotehyper.sty}{\usepackage{footnotehyper}}{\usepackage{footnote}}
\makesavenoteenv{longtable}
\usepackage{graphicx}
\makeatletter
\def\maxwidth{\ifdim\Gin@nat@width>\linewidth\linewidth\else\Gin@nat@width\fi}
\def\maxheight{\ifdim\Gin@nat@height>\textheight\textheight\else\Gin@nat@height\fi}
\makeatother
% Scale images if necessary, so that they will not overflow the page
% margins by default, and it is still possible to overwrite the defaults
% using explicit options in \includegraphics[width, height, ...]{}
\setkeys{Gin}{width=\maxwidth,height=\maxheight,keepaspectratio}
% Set default figure placement to htbp
\makeatletter
\def\fps@figure{htbp}
\makeatother

\KOMAoption{captions}{tableheading}
\makeatletter
\@ifpackageloaded{caption}{}{\usepackage{caption}}
\AtBeginDocument{%
\ifdefined\contentsname
  \renewcommand*\contentsname{Table of contents}
\else
  \newcommand\contentsname{Table of contents}
\fi
\ifdefined\listfigurename
  \renewcommand*\listfigurename{List of Figures}
\else
  \newcommand\listfigurename{List of Figures}
\fi
\ifdefined\listtablename
  \renewcommand*\listtablename{List of Tables}
\else
  \newcommand\listtablename{List of Tables}
\fi
\ifdefined\figurename
  \renewcommand*\figurename{Figure}
\else
  \newcommand\figurename{Figure}
\fi
\ifdefined\tablename
  \renewcommand*\tablename{Table}
\else
  \newcommand\tablename{Table}
\fi
}
\@ifpackageloaded{float}{}{\usepackage{float}}
\floatstyle{ruled}
\@ifundefined{c@chapter}{\newfloat{codelisting}{h}{lop}}{\newfloat{codelisting}{h}{lop}[chapter]}
\floatname{codelisting}{Listing}
\newcommand*\listoflistings{\listof{codelisting}{List of Listings}}
\makeatother
\makeatletter
\makeatother
\makeatletter
\@ifpackageloaded{caption}{}{\usepackage{caption}}
\@ifpackageloaded{subcaption}{}{\usepackage{subcaption}}
\makeatother
\ifLuaTeX
  \usepackage{selnolig}  % disable illegal ligatures
\fi
\usepackage{bookmark}

\IfFileExists{xurl.sty}{\usepackage{xurl}}{} % add URL line breaks if available
\urlstyle{same} % disable monospaced font for URLs
\hypersetup{
  pdftitle={Stuart Bowman \textbar{} Resume},
  colorlinks=true,
  linkcolor={blue},
  filecolor={Maroon},
  citecolor={Blue},
  urlcolor={Blue},
  pdfcreator={LaTeX via pandoc}}

\title{Stuart Bowman \textbar{} Resume}
\author{}
\date{}

\begin{document}
\maketitle

\renewcommand*\contentsname{Table of contents}
{
\hypersetup{linkcolor=}
\setcounter{tocdepth}{3}
\tableofcontents
}
I am:

\begin{itemize}
\tightlist
\item
  a proud member of the technical staff at the
  \href{https://www.mitre.org}{MITRE Corporation};
\item
  an aerospace engineer who loves writing scientific software for the
  aviation research community;
\item
  a \href{https://us-rse.org/}{research software engineer};
\item
  a developer of algorithms;
\item
  a researcher and modeler of Flight Management Systems;
\item
  a developer of scientific simulations;
\item
  a data analyst;
\item
  a technology fiddler.
\end{itemize}

I'm always interested in opportunities that:

\begin{itemize}
\tightlist
\item
  deepen my simulation expertise;
\item
  deepen my scientific expertise;
\item
  broadens my ability to talk about science and aviation;
\item
  uses scientific software thoughtfully for good science \&
  reproducibility;
\item
  pays attention to Open Science and open source software concepts.
\end{itemize}

Technology keywords:

\begin{itemize}
\tightlist
\item
  SOLID software development
\item
  C++, JAVA, Python3
\item
  Linux (Rocky, Raspberry Pi OS)
\item
  Aviation, Avionics, Flight Management System, Guidance, Navigation,
  Control, Robotics, Algorithms
\end{itemize}

\section{Research Engineer @ MITRE}\label{research-engineer-mitre}

My background is in the science of flight and that has been the main
focus of my work at MITRE: commercial aviation. The essential work of my
career has been to practice two deeply technical fields and bring them
together: aerospace engineering and software engineering. I do this by
building scientific simulations that enable scientifically defensible \&
reproducible research. I do this by staying close to both disciplines
simultaneously.

\section{Publications}\label{publications}

I enjoy writing about science; I mostly do it using computer languages.

\subsection{Open Source}\label{open-source}

At \href{https://www.mitre.org}{MITRE Corporation} I've been involved in
two major efforts that has produced important open-source code for the
civil aviation community. Take a look if you're so inclined.

\subsubsection{FIM MOPS Aircraft \& Control
Model}\label{fim-mops-aircraft-control-model}

\textbf{Link}: \url{https://mitre.github.io/FMACM/}

\textbf{Purpose}: A basic civil aviation simulation that contains
standard 3-DOF equations of motion, and basic control laws for a
guidance path. This code is used by avionics manufacturers when testing
for compliance to RTCA minimum operational standard
\href{https://my.rtca.org/NC__Product?id=a1B1R00000BdQlmUAF}{DO-361A}
and documented in this
\href{https://www.mitre.org/publications/technical-papers/derivation-of-a-point-mass-aircraft-model-used-for-fast-time}{MITRE
technical paper}.

\subsubsection{Interval Management Sample
Algorithm}\label{interval-management-sample-algorithm}

\textbf{Link}: \url{https://mitre.github.io/im_sample_algorithm/}

\textbf{Purpose}: A minimum ``sample'' implementation of the algorithms
necessary to comply with
\href{https://my.rtca.org/NC__Product?id=a1B1R00000BdQlmUAF}{DO-361A}.
This is used as a reference by manufacturers seeking to implement
Interval Management concepts in the Flight-deck.

\subsection{Conference, etc.}\label{conference-etc.}

\begin{itemize}
\tightlist
\item
  \emph{Analysis of the Use of Estimated Time of Arrival boradcast for
  Interval Management}. Air Traffic Management, 2017.
\item
  \emph{Closing the Loop: Testing for IM Avionics Certification}. AIAA
  SciTech, 2016.
\item
  \href{https://arc.aiaa.org/doi/abs/10.2514/2.1743}{\emph{Minimum Drag
  Power-Law Shapes for Rarefied Flow}}
\item
  \href{https://arc.aiaa.org/doi/abs/10.2514/2.3915}{\emph{Optimization
  of Low-Perigee Spacecraft Aerodynamics}}
\item
  Thesis: \emph{Numerical Optimization of Low-Perigee Spacecraft
  Shapes}, University of Maryland at College Park, May 2001.
\end{itemize}

\section{Outside MITRE}\label{outside-mitre}

\begin{itemize}
\tightlist
\item
  Robotics Instructor @
  \href{https://cornerstoneclassicalroanoke.org/faculty}{Cornerstone
  Classical Academy} in Roanoke, VA.
\item
  You can find out more about me at my blog:
  \href{https://www.aerosci.dev}{aerosci.dev}.
\end{itemize}

\section{Before MITRE}\label{before-mitre}

\textbf{8/2004- 2/2008} \textbar{} Systems Engineer with TOP SECRET
(SCI) clearance \textbar{} \href{https://www.saic.com/}{SAIC, Inc.},
Crystal City, VA

\begin{itemize}
\tightlist
\item
  Regularly perform maintenance and accreditation of physics-based,
  6-DoF simulations of the Tomahawk cruise missile as an integral member
  of the Mission Validation Analysis team.
\item
  Research performance concerns and prototype solutions, providing
  valuable support to the simulation development effort.
\item
  Develop software tools which increase the productivity of the team,
  including network resource management software and IDE-style scripting
  software.
\end{itemize}

\textbf{7/2003 -- 8/2004} \textbar{} Systems Engineer \textbar{}
\href{https://en.wikipedia.org/wiki/SPARTA,_Inc.}{SPARTA, Inc.},
Rosslyn, VA

\begin{itemize}
\tightlist
\item
  Interacted with the Missile Defense National Team, producing documents
  and presentations which advised the government regarding existing and
  developing ballistic missile threats around the world.
\end{itemize}

\textbf{7/2001 -- 7/2003} \textbar{} Engineer \textbar{} Systems
Engineering Group, Inc., Columbia, MD

\begin{itemize}
\tightlist
\item
  Employed knowledge of aerodynamics, propulsion, and flight dynamics in
  the development of classified and unclassified missile models, and
  used those models to generate reliable data for our Naval customer.
\item
  Maintained and streamlined a six degree-of-freedom point-mass
  simulation code, providing valuable support and increased
  functionality for the users.
\item
  Conducted research in areas key to the missile modeling process and
  produced memorandums which accurately summarized the results.
\end{itemize}

\textbf{6/1999 - 6/2001} \textbar{} Graduate Research Assistant
\textbar{} University of Maryland College Park \textasciitilde{} Center
for Hypersonic Education and Research, College Park, MD

\begin{itemize}
\tightlist
\item
  Developed and designed a low-perigee spacecraft geometry model and an
  analytical, rarefied flow model resulting in software which calculated
  a spacecraft's on- and off-design forces and moments.
\item
  Integrated the geometry and aerodynamic flow models with a numerical,
  gradient-based optimizer and successfully explored the spacecraft's
  design space, producing a reduced drag spacecraft shape for NASA
  Goddard's Geospace Electrodynamics Connections mission.
\end{itemize}

\section{Education}\label{education}

2001 \textbar{} Master of Science \textbar{} University Of Maryland
College Park

Concentration: Aerospace Engineering Thesis: Numerical Optimization of
Low-Perigee Spacecraft Shapes Key Courses: Astrodynamics,
High-Temperature Gas Dynamics, Viscous Flow, High-Speed Propulsion,
Burning Theory, Computational Fluid Dynamics, Engineering Optimization,
Electric Propulsion 1999 \textbar{} Bachelor of Science \textbar{}
University Of Maryland College Park

Concentration: Aerospace Engineering, Graduated Magna Cum Laude
(3.86/4.0) Key Courses: Aerodynamics (Hypersonic, Compressible,
Incompressible), Air-breathing Propulsion, Aircraft Performance and
Control, Aircraft Design, Aircraft Structures, Technical Writing



\end{document}
